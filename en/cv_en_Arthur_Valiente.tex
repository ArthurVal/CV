%%%%%%%%%%%%%%%%%%%%%%%%%%%%%%%%%%%%%%%%%
% Twenty Seconds Resume/CV
% LaTeX Template
% Version 1.0 (14/7/16)
%
% Original author:
% Carmine Spagnuolo (cspagnuolo@unisa.it) with major modifications by
% Vel (vel@LaTeXTemplates.com) and Harsh (harsh.gadgil@gmail.com)
%
% License:
% The MIT License (see included LICENSE file)
%
% --------------------------------------
% Notes Arthur
%
% Prérequis:
% -- Pkg:
% -> xetex
% -> fonts-extra
% -> pstricks
%
%%%%%%%%%%%%%%%%%%%%%%%%%%%%%%%%%%%%%%%%%

%----------------------------------------------------------------------------------------
%	PACKAGES AND OTHER DOCUMENT CONFIGURATIONS
%----------------------------------------------------------------------------------------


\documentclass[letterpaper]{twentysecondcv} % a4paper for A4

% Command for printing skill overview bubbles
\newcommand\skills{

  \smartdiagram[bubble diagram]{
    \textbf{\large Engineer}\\\textbf{\large Embedded}\\\textbf{\large Robotic},
    \textbf{\large \ros}\\\textbf{Middleware},
    \textbf{\large \href{https://en.wikipedia.org/wiki/Simultaneous_localization_and_mapping}{S.L.A.M.}},
    \textbf{\large Kalman}\\\textbf{\href{https://en.wikipedia.org/wiki/Extended_Kalman_filter}{EKF} | \href{https://en.wikipedia.org/wiki/Kalman_filter#Unscented_Kalman_filter}{UKF}},
    \textbf{\large Embedded}\\\textbf{ARM-DSP},
    \textbf{~~\huge \faLinux~~}\\\textbf{\large \href{https://www.yoctoproject.org/}{Yocto}},
    \textbf{\large Optimal}\\\textbf{\large Control}
  }
}
  
\newcommand{\cea}{\href{http://www.cea.fr/english}{CEA}}
\newcommand{\ros}{\href{http://www.ros.org/}{R.O.S.}}
  
% Programming skill bars (0->6)
\programming{
  {JavaScript $\textbullet$ \large \LaTeX / 1},
  {Matlab $\textbullet$ Simulink / 3},
  {Bash $\textbullet$ CMake $\textbullet$ Make $\textbullet$ Emacs / 3.5},
  {C $\textbullet$ C++  $\textbullet$ Python $\textbullet$ Git $\textbullet$ \ros / 5}}


% Lang skill bars (0->6)
\lang{
  {Spanish $\textbullet$ Japanese / 0.5},  
  {English / 4.5},
  {French / 6}}

%----------------------------------------------------------------------------------------
%	 PERSONAL INFORMATION
%----------------------------------------------------------------------------------------
% If you don't need one or more of the below, just remove the content leaving the command, e.g. \cvnumberphone{}

\cvname{Arthur Valiente} % Your name
\cvjobtitle{ \textbf{Research Engineer} \\ Robotic - Automation \\ Embedded Systems} % Job
% title/career

\cvlinkedin{/in/arthur-valiente-1866937b}
\cvgithub{ArthurVal}
\cvnumberphone{+33.6.77.20.10.54} % Phone number
\cvsite{} % Personal website
\cvaddress{\parbox[c]{5cm}{32 Rue Charles de Gaulle \newline 91400 Orsay - France}}
\cvmail{valiente.arthur@gmail.com} % Email address

%----------------------------------------------------------------------------------------

\begin{document}

\makeprofile % Print the sidebar

%----------------------------------------------------------------------------------------
%	 EDUCATION
%----------------------------------------------------------------------------------------
\section{Education}
\begin{twenty}
  \twentyitem
  {2012 - 2015}
  {}
  {Master degree in Engineering: Electrical \& Automation}
  {\href{http://www.enseeiht.fr/en/index.html}{INP-ENSEEIHT}}
  {
    \vspace{-2mm}               %Hack because I don't know why but there is
                                %blank space ..
    \begin{itemize}
    \item \small Control, Decision and IT for Critical Systems (CDISC)
    \item \small Development of Critical IT Systems (DeSIC)
    \end{itemize}
  }
  \\
  \twentyitem
  {2010 - 2012}
  {}
  {2-years University degree in technology (HND)}
  {\href{http://iut-meph.ups-tlse.fr/}{I.U.T Paul Sabatier, Toulouse}}
  {
    \vspace{-2mm}
    \begin{itemize}
    \item \small Physicals Measurements
    \item \small Instrumental Techniques
    \end{itemize}
  }
  \\
  \twentyitem
  {2010}
  {}
  {A-level in Engineer Science}
  {\href{http://alexis-monteil.entmip.fr/}{Lycée Alexis Monteil, Rodez}}
  {}
  % \twentyitem{<dates>}{<dates>}{<title>}{<location>}{<description>}
\end{twenty}
%----------------------------------------------------------------------------------------
%	 EXPERIENCE
%----------------------------------------------------------------------------------------
\section{Experience}
\begin{twenty} % Environment for a list with descriptions
  \twentyitem
  {Jul 2018 -}
  {Now}
  {Fixed-term contract: Research Engineer, 18 months}
  {\cea-DRF-IRFU, Paris}
  {
    \textbf{\href{http://www.svom.fr/en/}{SVOM Project}}: In charge of the
    automatic test bench of the onboard software for the ECLAIRS Gamma-Ray instrument
    \begin{itemize}
    \item \textbf{\cea\ - \href{http://www.irap.omp.eu/en/}{IRAP} -
        \href{https://cnes.fr/en}{CNES}} collaboration
    \item Development of the distributed Local Test Unit (2x Zedboards, PC, ...)
      (C++, Python)
    \item \href{https://docs.pytest.org/en/latest/contents.html}{PyTest}
      testing framework development
    \item Development of scripts associated to the unit test case (Python)
    \end{itemize}
  }
  \\
  \twentyitem
  {Jan 2017 -}
  {Dec 2017}
  {Fixed-term contract: Research Engineer, 12 months}
  {\cea-DRT-LIST, Paris}
  {
    In charge of the software development for robotic applications
    \begin{itemize}
    \item Co-responsible of the software development (Git Admin - \ros)
    \item \href{https://en.wikipedia.org/wiki/Voronoi_diagram}{Voronoi diagram}
      \& Occupancy grid algorithms development (C++)
    \item Hybrid topological SLAM for mobile robot (Lidar / IMU / RBG-D Camera /
      ...) (C++)
    \item
      \textbf{\href{https://www.youtube.com/watch?time_continue=3&v=mneZNnir0s0}{FACE
        Project}}: In charge of the multi-\href{https://en.wikipedia.org/wiki/System_on_a_chip}{SoC} prototype (Renesas
      R-Car H3, Nvidia Tx3, Kalray MPPA Bostan)
    \item Linux Yocto layers development for R-Car H3
    \end{itemize}
  }
  \\
  \twentyitem
  {Sep 2015 -}
  {Dec 2015}
  {Fixed-term contract: Research Engineer, 3 months}  
  {\href{https://www.irit.fr/?lang=en}{IRIT, Toulouse}}
  {
    \textbf{\href{http://www.agence-nationale-recherche.fr/Project-ANR-12-CORD-0003}{ANR
        RIDDLE} Project}: Development of the robotic demonstration on
    \href{http://www.willowgarage.com/pages/pr2/overview}{PR2} and
    \href{https://spectrum.ieee.org/automaton/robotics/humanoids/aldebaran-robotics-introduces-romeo-finally}{ROMEO} robots
    \begin{itemize}
    \item Importing object recognition and human interaction (vision and voice)
      algorithms on
      \href{https://spectrum.ieee.org/automaton/robotics/humanoids/aldebaran-robotics-introduces-romeo-finally}{ROMEO}
      robot of Aldebaran Robotics
    \item Creating a bridge between object recognition algorithms - simulator
      using \ros\ (C++)
    \end{itemize}
  }
  \\
  \twentyitem
  {Mar 2015 -}
  {Sep 2015}
  {Internship: Computer Vision data fusion}
  {\href{https://www.laas.fr/public/en}{CNRS-LAAS, Toulouse}}
  {
    \textbf{\href{http://www.agence-nationale-recherche.fr/Project-ANR-12-CORD-0003}{ANR
        RIDDLE} Project}: Fusion of computer vision \& radio-frequency method for
    object recognition
    \begin{itemize}
    \item Development of object recognition software on \ros\ using RGB-D
      Kinect camera
      (\href{http://www.stefan-hinterstoisser.com/papers/hinterstoisser2011linemod.pdf}{LINEMOD}
      \& \href{http://wiki.ros.org/tabletop_object_detector}{TABLETOP}) (C++)
    \item Creating a custom communication protocol using
      \href{https://en.wikipedia.org/wiki/Universal_asynchronous_receiver-transmitter}{UART}
      on \href{https://en.wikipedia.org/wiki/Digital_signal_processor}{DSP} (C, C++)
    \end{itemize}
  }
  \\
  \twentyitem
  {Feb 2015 -}
  {Mar 2015}
  {Student Project: Sound sources localization}
  {\href{https://www.laas.fr/public/en}{CNRS-LAAS, Toulouse}}
  {
    \textbf{\href{http://twoears.eu/}{TWO!EARS} Project}: Probabilistic
    algorithms for binaural sound sources localization
    \begin{itemize}
    \item Multi-Hypothesis Unscented Kalman Filter (MH-UKF) (C++)
    \item Binaural sound sources localization on \ros\ (C++)
    \end{itemize}
  }
  \\
  \twentyitem
  {Jul 2014 -}
  {Aug 2014}
  {Internship: Technician}
  {\href{http://www.spherea.com/en}{Cassidian Test \& Services, Colomiers}}
  {
    Technician Internship - PCIe link on FPGA (VHDL)
  }
  \\
  \twentyitem
  {Jul 2013 -}
  {Aug 2013}
  {Intership: Worker}
  {Tokyo / Kitakyushu, Japon}
  {
    House painter
  }
  \\
  \twentyitem
  {Apr 2012 -}
  {Jun 2012}
  {Internship: Technician}  
  {\href{https://www.gtptech.com/}{GTP Technology}, Labège}
  {
    Instrumentation technician - Data acquisition (Modbus, UART, RS-485)
  }  
  % \twentyitem{<dates>}{<title>}{<location>}{<description>}
\end{twenty}
\end{document}

%%% Local Variables:
%%% mode: LaTeX
%%% TeX-PDF-mode: t
%%% TeX-command-extra-options: "-output-directory=out"
%%% TeX-engine: xetex
%%% End:
%%%%%%%%%%%%%%%%%%%%%%%%%%%%%%%%%%%%%%%%%
% Twenty Seconds Resume/CV
% LaTeX Template
% Version 1.0 (14/7/16)
%
% Original author:
% Carmine Spagnuolo (cspagnuolo@unisa.it) with major modifications by
% Vel (vel@LaTeXTemplates.com) and Harsh (harsh.gadgil@gmail.com)
%
% License:
% The MIT License (see included LICENSE file)
%
% --------------------------------------
% Notes Arthur
%
% Prérequis:
% -- Pkg:
% -> xetex
% -> fonts-extra
% -> pstricks
%
%%%%%%%%%%%%%%%%%%%%%%%%%%%%%%%%%%%%%%%%%

%----------------------------------------------------------------------------------------
%	PACKAGES AND OTHER DOCUMENT CONFIGURATIONS
%----------------------------------------------------------------------------------------

\documentclass[letterpaper]{twentysecondcv} % a4paper for A4

% Command for printing skill overview bubbles
\newcommand\skills{
  ~
  \smartdiagram[bubble diagram]{
    \textbf{\large Ingénieur}\\\textbf{\large Robotique},
    \textbf{\large \ros}\\\textbf{Intergiciel},
    \textbf{\large \href{https://en.wikipedia.org/wiki/Simultaneous_localization_and_mapping}{S.L.A.M.}},
    \textbf{\large Kalman}\\\textbf{\href{https://en.wikipedia.org/wiki/Extended_Kalman_filter}{EKF} | \href{https://en.wikipedia.org/wiki/Kalman_filter#Unscented_Kalman_filter}{UKF}},
    \textbf{\large Embarqué}\\\textbf{ARM-DSP},
    \textbf{~~\huge \faLinux~~}\\\textbf{\large \href{https://www.yoctoproject.org/}{Yocto}},
    \textbf{Traitement}\\\textbf{Image}
  }
}
  
\newcommand{\cea}{\href{http://www.cea.fr/}{CEA}}
\newcommand{\ros}{\href{http://www.ros.org/}{R.O.S.}}
  
% Programming skill bars (0->6)
\programming{
  {JavaScript $\textbullet$ \large \LaTeX / 1},
  {Matlab $\textbullet$ Simulink / 3},
  {Bash $\textbullet$ CMake $\textbullet$ Make $\textbullet$ Emacs / 3.5},
  {C $\textbullet$ C++  $\textbullet$ Python $\textbullet$ Git $\textbullet$ \ros / 5}}


% Lang skill bars (0->6)
\lang{
  {Espagnol $\textbullet$ Japonais / 0.5},  
  {Anglais / 4.5},
  {Français / 6}}

%----------------------------------------------------------------------------------------
%	 PERSONAL INFORMATION
%----------------------------------------------------------------------------------------
% If you don't need one or more of the below, just remove the content leaving the command, e.g. \cvnumberphone{}

\cvname{Arthur Valiente} % Your name
\cvjobtitle{ \textbf{Ingénieur INP-ENSEEIHT} \\ Robotique - Automatique \\ Systèmes Embarqués} % Job
% title/career

\cvlinkedin{/in/arthur-valiente-1866937b}
\cvgithub{ArthurVal}
\cvnumberphone{+33.6.77.20.10.54} % Phone number
\cvsite{} % Personal website
\cvaddress{\parbox[c]{5cm}{32 Rue Charles de Gaulle \newline 91400 Orsay - France}}
\cvmail{valiente.arthur@gmail.com} % Email address

%----------------------------------------------------------------------------------------

\begin{document}

\makeprofile % Print the sidebar

%----------------------------------------------------------------------------------------
%	 EDUCATION
%----------------------------------------------------------------------------------------
\section{Formation}
\begin{twenty}
  \twentyitem
  {2012 - }
  {2015}
  {Diplôme Ingénieur: Génie Électrique et Automatique}
  {\href{http://www.enseeiht.fr/fr/index.html}{INP-ENSEEIHT}}
  {}
  {
    \begin{itemize}
    \item \small Commande, Décision et Informatique des Systèmes Critiques (CDISC)
    \item \small Développement des Systèmes Informatiques Critiques (DeSIC)
    \end{itemize}
  }
  \\
  \twentyitem
  {2010 - }
  {2012}
  {D.U.T. Mesures Physiques}
  {\href{http://iut-meph.ups-tlse.fr/}{I.U.T Paul Sabatier, Toulouse}}
  {}
  {
    \begin{itemize}
    \item \small Techniques Instrumentales
    \end{itemize}
  }
  \\
  \twentyitem
  {2010}
  {}
  {Baccalauréat S - Science de l'Ingénieur}
  {\href{http://alexis-monteil.entmip.fr/}{Lycée Alexis Monteil, Rodez}}
  {}
  {
  }
  % \twentyitem{<dates>}{<title>}{<location>}{<description>}
\end{twenty}

%----------------------------------------------------------------------------------------
%	 EXPERIENCE
%----------------------------------------------------------------------------------------
\section{Expérience}
\begin{twenty} % Environment for a list with descriptions
  \twentyitem
  {Juil 2018 -}
  {Présent}
  {CDD: Ingénieur Recherche et Développement}
  {\cea-DRF-IRFU, Saclay}
  {}
  {
    Responsable du banc de test logiciel embarqué caméra ECLAIRs
    \begin{itemize}
    \item \textbf{\href{http://www.svom.fr/}{Projet SVOM}} - Collaboration
      \textbf{\cea - \href{http://www.irap.omp.eu/}{IRAP} - \href{https://cnes.fr/fr}{CNES} }
    \item Développement C++/Python de l'infrastructure de test globale
    \item Utilisation séquenceur de test \href{https://docs.pytest.org/en/latest/contents.html}{PyTest}
    \item En charge de la création des scripts de tests unitaires Python
    \end{itemize}
  }
  \\
  \twentyitem
  {2017}
  {}
  {CDD: Ingénieur Recherche et Développement}
  {\cea-DRT-LIST, Saclay}
  {}
  {
    Responsable développements logiciel pour applications robotique
    \begin{itemize}
    \item Co-responsable développement logiciel (Git Admin - \ros)
    \item Importation algorithme SLAM hybride topologique pour robot mobile (Lidar / IMU / Caméra RBG-D / ...)
    \item Création d'applications C++ pour prototype
      multi-\href{https://en.wikipedia.org/wiki/System_on_a_chip}{SoC} (Renesas R-Car
      H3, Nvidia Tx3, Kalray MPPA Bostan)
    \item Responsable intégration logicielle prototype
      multi-\href{https://en.wikipedia.org/wiki/System_on_a_chip}{SoC}
    \item Développement 'layers' Linux Yocto pour R-Car H3
    \end{itemize}
  }
  \\
  \twentyitem
  {Sept 2015 -}
  {Déce 2015}
  {CDD: Ingénieur d'Étude}
  {\href{https://www.irit.fr/?lang=fr}{IRIT, Toulouse}}
  {}
  {
    \textbf{Projet
      \href{http://www.agence-nationale-recherche.fr/Project-ANR-12-CORD-0003}{ANR
        RIDDLE}} : Développement démonstration robotique sur robots \href{http://www.willowgarage.com/pages/pr2/overview}{PR2} et \href{https://spectrum.ieee.org/automaton/robotics/humanoids/aldebaran-robotics-introduces-romeo-finally}{ROMEO}
    \begin{itemize}
    \item Importation d'algoritmes de détection d'objets, intéractions (vision
      et vocal) sur robot
      \href{https://spectrum.ieee.org/automaton/robotics/humanoids/aldebaran-robotics-introduces-romeo-finally}{ROMEO}
      d'Aldebaran Robotics
    \item Création d'interfaces détecteurs d'objets / simulateurs via \ros
    \end{itemize}
  }
  \\
  \twentyitem
  {Mars 2015 -}
  {Sept 2015}
  {Stage: Projet de Fin d'Étude - Ingénieur}
  {\href{https://www.laas.fr/public/fr}{CNRS-LAAS, Toulouse}}
  {}
  {
    \textbf{Projet
      \href{http://www.agence-nationale-recherche.fr/Project-ANR-12-CORD-0003}{ANR
        RIDDLE}} : Fusion traitement d'image / détecteur radiofréquence pour
    détection d'objets
    \begin{itemize}
    \item Développement C++ sur \ros de briques
      détection d'objets via traitement d'image
      (\href{http://www.stefan-hinterstoisser.com/papers/hinterstoisser2011linemod.pdf}{LINEMOD}
      \& \href{http://wiki.ros.org/tabletop_object_detector}{TABLETOP})
    \item Développement protocole de communication C++ via
      \href{https://en.wikipedia.org/wiki/Universal_asynchronous_receiver-transmitter}{UART}
      sur \href{https://en.wikipedia.org/wiki/Digital_signal_processor}{DSP}
    \end{itemize}
  }
  \\
  \twentyitem
  {Fevr 2015 -}
  {Mars 2015}
  {Projet Étudiant: Projet Long - Ingénieur}
  {\href{https://www.laas.fr/public/fr}{CNRS-LAAS, Toulouse}}
  {}
  {
    \textbf{Projet
      \href{http://twoears.eu/}{TWO!EARS}} : Algorithme probabilistique de
    localisation binaurale
    \begin{itemize}
    \item Développement C++ de filtres de Kalman Unscented Multi-Hypothèses
      (MH-UKF)
    \item Détection de source sonore binaurale sur \ros
    \end{itemize}
  }
  \\
  \twentyitem
  {Juil 2014 -}
  {Août 2014}
  {Stage: Technicien}
  {\href{http://www.spherea.com/fr}{Cassidian Test \& Services, Colomiers}}
  {}
  {
    Stage technicien création liaison PCIe VHDL sur FPGA
  }
  \\
  \twentyitem
  {Juil 2013 -}
  {Août 2013}
  {Stage: Ouvrier}
  {Tokyo / Kitakyushu, Japon}
  {}
  {
    Ouvrier peintre en bâtiment
  }
  \\
  \twentyitem
  {Avri 2012 -}
  {Juin 2012}
  {Stage: Technicien}
  {\href{https://www.gtptech.com/}{GTP Technology}, Labège}
  {}
  {
    Technicien en instrumentation, acquisition de données (Modbus, UART, RS-485)
  }  
  % \twentyitem{<dates>}{<title>}{<location>}{<description>}
\end{twenty}
\end{document}

%%% Local Variables:
%%% mode: LaTeX
%%% TeX-PDF-mode: t
%%% TeX-command-extra-options: "-output-directory=out"
%%% TeX-engine: xetex
%%% End: